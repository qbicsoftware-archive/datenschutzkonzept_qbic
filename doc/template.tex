\documentclass[]{scrreprt}
\usepackage[utf8]{inputenc}
\usepackage{graphicx}
\usepackage{ngerman}

\begin{document}
\begin{titlepage}
\begin{center}
	\includegraphics[width=0.4\textwidth]{../figs/qbic_logo.png} \\
	\vspace{2cm}
	{\LARGE Datenschutz- und Datensicherheitskonzept \\
		am \\
		Zentrum für Quantitative Biologie\\[2cm]}
	
	{\large{Fassung vom 24. April 2017}\\[1cm]} %% alternativ: \today
	\vfill
	{\large Dr. Sven Nahnsen, Matthias Seybold, Sven Fillinger}
	
\end{center}
\begin{abstract}
	Hier kann man einen allgemeinen Einführungstext schreiben, zum Beispiel Motivation, Gesetzgebung und Ziele der Datensicherheitskonzepte am QBiC.
\end{abstract}
\end{titlepage}

\section*{Änderungsnachweis}
\subsection*{3. Juli 2014}
{\small\textbf{Author(en): Dr. Sven Nahnsen}}\\
Neuerstellung und erste Fassung des Datenschutz- und Datensicherheitskonzepts

\subsection*{24. Juli 2017}
{\small\textbf{Author(en): Dr. Sven Nahnsen, Matthias Seybold, Sven Fillinger}}\\
Layout Neuauflage, Aktualisierung von technischen Daten, Überarbeitung der Begriffsabgrenzung \texttt{Datenschutz} und \texttt{Datensicherheit}.
\newpage

\tableofcontents{}

\chapter*{Betrieb des Zentrums für Quantitative Biologie (QBiC)}
\addcontentsline{toc}{chapter}{Betrieb des Zentrums für Quantitative Biologie (QBiC)}%
\section*{Präambel}
\addcontentsline{toc}{section}{Präambel}%
Das Quantitative Biology Center (QBiC) wurde 2011 als zentrale Einrichtung der Universität Tübingen und mit Zusammenarbeit der Medizinischen Fakultät Tübingen, sowie dem Max Planck Instituts für Entwicklungsbiologie gegründet. Der technische Betrieb von der QBiC Infrastruktur findet dabei am Zentrum für Datenverarbeitung (ZDV) der Eberhard Karls Universität Tübingen statt.

Dieses Handbuch beschreibt die konkrete Umsetzung der in § 9 Landesdatenschutzgesetz (LDSG) geforderten technischen und organisatorischen Maßnahmen zum Datenschutz bei automatiserter Verarbeitung personenbezogener Daten und zurdas  Datensicherheitskonzept am QBiC.

Die nachfolgenden Ausführungen untergliedern sich in Erläuterungen zu datenschutzrechtlichen Begriffen sowie der Definition der Schutzmaßnahmen für die zentrale DV-Anlage, den Netzwerkkomponenten für den Zugang und den lokalen Anschluss. 


\chapter{Grundlagen eines Datenschutzkonzeptes gemäß LDSG}
Die in diesem Datenschutzkonzept festgeschriebenen Maßnahmen sollen den Missbrauch und die Verfälschung von  personenbezogenen Daten verhindern. Werden in öffentlichen Stellen selbst oder im Auftrag Daten dieser Art verarbeitet, so haben die für die Datenverarbeitung (DV) verantwortlichen Stellen gemäß § 9 (3) LDSG in der Fassung vom 2.April 2003 (GBl. S 648)  technische und organisatorische Maßnahmen zu treffen, um die Ausführung der Vorschriften dieses Gesetzes zu gewährleisten.

\noindent
So ist zu gewährleisten, dass
\begin{itemize}
	\item nur berechtigte Personen auf Datenbestände Zugriff haben (Vertraulichkeit),
	\item Daten bei der Verarbeitung nicht verfälscht werden (Integrität),
	\item Datenbestände reproduziert werden können (Verfügbarkeit).
\end{itemize}

\noindent
Diese Ziele können erreicht werden durch:
\begin{itemize}
	\item gebäudetechnische Maßnahmen (gebäudespezifische und räumliche Absicherung),
	\item hardwaretechnische Maßnahmen (Hardware-Passwortschutz, Schlösser etc.),
	\item softwaretechnische Maßnahmen (Software-Passwortschutz, Auditing etc.).
\end{itemize}

Aus o. g. Gesetzestext geht weiter hervor, dass solche Maßnahmen nur erforderlich sind, wenn „[...]der Aufwand, insbesondere unter Berücksichtigung der Art der zu schützenden Daten, in einem angemessenen Verhältnis zum Schutzzweck steht“(siehe LDSG §9 (2)).
Es wird daher nachfolgend ein Konzept erarbeitet, welches ausgehend von den am meisten zu schützenden Datenbeständen einen für das gesamte DV System ausreichenden Grundschutz gewährleistet.


\section{Definition der Maßnahmen gemäß § 9 LDSG}
Werden personenbezogene Daten automatisiert verarbeitet, sind Maßnahmen zu treffen, welche die in den folgenden Kapiteln definierten Kontrollvorgaben berücksichtigen.
\subsection{§ 9 (3) Nr. 1 : Zutrittskontrolle}
Im Rahmen der Zutrittskontrolle ist Unbefugten der Zugang zu Datenverarbeitungsanlagen zu verwehren. Geeignete Maßnahmen sind dedizierte EDV- und Verteilerräume, verschlossene Diensträume sowie restriktive Schlüsselvergabe.
\subsection{§ 9 (3) Nr. 2 : Datenträgerkontrolle}
Die Maßnahmen zur Datenträgerkontrolle sollen verhindern, dass Datenträger unbefugt gelesen, kopiert, verändert oder entfernt werden können.
\subsection{§ 9 (3) Nr. 3 : Speicherkontrolle}
Die Speicherkontrolle dient der Vermeidung von unbefugten Eingaben in den Speicher sowie Verhinderung unbefugter Kenntnisnahme, Veränderung oder Löschung gespeicherter Daten.
\subsection{§ 9 (3) Nr. 4 : Benutzerkontrolle}
Wird durch technische oder organisatorische Maßnahmen verhindert, dass Datenverarbeitungsanlagen (DVA) mit Hilfe von Einrichtungen zur Datenübertragung unberechtigt genutzt werden, so spricht man von Benutzerkontrolle.
\subsection{§ 9 (3) Nr. 5 : Zugriffskontrolle}
Die Zugriffskontrolle gewährleistet, dass die zur Benutzung einer DVA Berechtigten ausschließlich auf die ihrer Zugriffsberechtigung unterliegenden Daten zugreifen können.
\subsection{§ 9 (3) Nr. 6 : Übermittlungskontrolle}
Ziel der Übermittlungskontrolle ist es, zu gewährleisten, dass überprüft und festgestellt werden kann, an welchen Stellen Daten durch Einrichtungen zur Datenübertragung übermittelt werden können.
\subsection{§ 9 (3) Nr. 7 : Eingabekontrolle}
Die Eingabekontrolle stellt sicher, dass nachträglich überprüft und festgestellt werden kann, welche Daten zu welcher Zeit von wem in ein Datenverarbeitungssystem (DVS) eingegeben worden sind.
\subsection{§ 9 (3) Nr. 8 : Auftragskontrolle}
Die Auftragskontrolle impliziert, dass eine Verarbeitung von Daten im Auftrag nur entsprechend den Weisungen des Auftraggebers durchgeführt werden kann.
\subsection{§ 9 (3) Nr. 9 : Transportkontrolle}
Ziel der Transportkontrolle ist es, zu verhindern, dass sowohl bei der Übertragung als auch während des Transports von Daten auf Datenträgern diese nicht unbefugt gelesen, kopiert, verändert oder gelöscht werden können. 
\subsection{§ 9 (3) Nr. 10 : Verfügbarkeitskontrolle}
Die Verfügbarkeitskontrolle soll gewährleisten, dass personenbezogene Daten gegen Zerstörung oder Verlust geschützt sind.
\subsection{§ 9 (3) Nr. 11 : Organisationskontrolle}
Organisationskontrolle bedeutet, dass innerbehördliche und innerbetriebliche Organisationen so zu gestalten sind, dass sie den besonderen Anforderungen des Datenschutzes gerecht werden.
Nachfolgend der Maßnahmenkatalog für den Betrieb von QBiC, differenziert nach zentralen Netzwerkkomponenten und Arbeitsstationen.


\end{document}          
